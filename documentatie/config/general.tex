%   Auto formatting
\usepackage[top=1in, bottom=1in, left=1in, right=1in]{geometry}
\usepackage{sidecap,wrapfig}
\usepackage{fancyhdr}
\usepackage{parskip}
\usepackage{varwidth}
%\usepackage{float}
%\usepackage{tocloft}

%   Tables
\usepackage{longtable}
\usepackage{multicol}
\usepackage{tabularx}
\usepackage{ltablex}

%   Lijsten
\usepackage{tasks}

%   Formules
\usepackage{amsmath}


\usepackage[tablegrid]{vhistory}    % Versiehistorie
\usepackage{datetime}   % Custom datums

%   Taalvoorkeur
\usepackage[dutch]{babel}

%   Lettertype aanpassen
\usepackage[T1]{fontenc}
\usepackage[utf8]{inputenc}
\usepackage{charter}
\usepackage[gen]{eurosym}

%ganttchart
\usepackage{pgfgantt}

%   Opmaak voor code
\usepackage{listings}
\usepackage{xcolor}

%   Zorgt voor correcte referenties 
\usepackage[hidelinks,bookmarksnumbered]{hyperref}
\usepackage{fancyref}
\usepackage{nameref}
\usepackage{lastpage}
\usepackage[dutch,noabbrev]{cleveref}
\usepackage{bookmark}
\usepackage[nottoc]{tocbibind}
\usepackage[backend=bibtex,style=ieee,sorting=ynt,autocite=superscript]{biblatex}
\usepackage{csquotes}

%   Importeer afbeeldingen en documenten
%\usepackage{pdfpages}
\usepackage{svg}
\usepackage{graphics}
\usepackage{graphicx}
\usepackage{epstopdf}
\usepackage{tikz}
\usepackage[europeanvoltages,europeancurrents,europeanresistors]{circuitikz}
\usetikzlibrary{arrows,positioning,shapes.geometric}

\usepackage{subcaption}
\usepackage{blindtext}

\makeatletter
\pgfdeclareshape{datastore}{
  \inheritsavedanchors[from=rectangle]
  \inheritanchorborder[from=rectangle]
  \inheritanchor[from=rectangle]{center}
  \inheritanchor[from=rectangle]{base}
  \inheritanchor[from=rectangle]{north}
  \inheritanchor[from=rectangle]{north east}
  \inheritanchor[from=rectangle]{east}
  \inheritanchor[from=rectangle]{south east}
  \inheritanchor[from=rectangle]{south}
  \inheritanchor[from=rectangle]{south west}
  \inheritanchor[from=rectangle]{west}
  \inheritanchor[from=rectangle]{north west}
  \backgroundpath{
    %  store lower right in xa/ya and upper right in xb/yb
    \southwest \pgf@xa=\pgf@x \pgf@ya=\pgf@y
    \northeast \pgf@xb=\pgf@x \pgf@yb=\pgf@y
    \pgfpathmoveto{\pgfpoint{\pgf@xa}{\pgf@ya}}
    \pgfpathlineto{\pgfpoint{\pgf@xb}{\pgf@ya}}
    \pgfpathmoveto{\pgfpoint{\pgf@xa}{\pgf@yb}}
    \pgfpathlineto{\pgfpoint{\pgf@xb}{\pgf@yb}}
 }
}
\makeatother

\tikzset{
  font=\sffamily,
  every matrix/.style={ampersand replacement=\&,column sep=2cm,row sep=2cm},
  blank/.style={},
  terminator/.style={draw,thick,inner sep=.3cm,minimum width=3cm,minimum height=2cm},
  process/.style={draw,thick,circle,minimum size=3cm},
  cprocess/.style={draw,thick,circle,minimum size=3cm, dashed},
  datastore/.style={draw,very thick,shape=datastore,inner sep=.3cm, minimum width=3cm,minimum height=1cm},
  dots/.style={gray,scale=2},
  flow/.style={->,>=stealth',shorten >=1pt,semithick,font=\sffamily\footnotesize},
  cflow/.style={->,>=stealth',shorten >=1pt,semithick,font=\sffamily\footnotesize, dashed},
  every node/.style={align=center},
  block/.style= {draw, rounded corners, rectangle, align=center,minimum width=3cm,minimum height=2cm},
  module/.style={draw,thick,inner sep=.3cm,minimum width=3cm,minimum height=2cm, rounded corners}
}

\newcommand{\version}{1.0}

\newcommand{\setpagestyle}{
    \fancyhf{}
    \fancyfoot[LE,LO]{Version \version}
    \fancyfoot[RE,RO]{Page \thepage/\pageref{LastPage}}
    \renewcommand{\headrulewidth}{0pt}% Line at the header invisible
    \renewcommand{\footrulewidth}{0.4pt}% Line at the footer visible
}
\pagestyle{fancy}

\fancypagestyle{plain}{
    \setpagestyle
}
\setpagestyle

\newcommand{\authorcite}[1]{\citeauthor{#1}\,\supercite{#1}}

\usetikzlibrary{shapes.geometric, arrows}
\tikzstyle{arrow} = [thick,->,>=stealth]
\tikzstyle{startstop} = [rectangle, rounded corners, minimum width=3cm, minimum height=1cm,text centered, draw=black]
\tikzstyle{io} = [trapezium, trapezium left angle=70, trapezium right angle=110, minimum width=3cm, minimum height=1cm, text centered, draw=black]
\tikzstyle{process} = [rectangle, minimum width=3cm, minimum height=1cm, text centered, draw=black]
\tikzstyle{decision} = [diamond, minimum width=3cm, minimum height=1cm, text centered, draw=black]

\lstset{
  language=C,                % choose the language of the code
  numbers=left,                   % where to put the line-numbers
  stepnumber=1,                   % the step between two line-numbers.        
  numbersep=5pt,                  % how far the line-numbers are from the code
  showspaces=false,               % show spaces adding particular underscores
  showstringspaces=false,         % underline spaces within strings
  showtabs=false,                 % show tabs within strings adding particular underscores
  tabsize=2,                      % sets default tabsize to 2 spaces
  captionpos=b,                   % sets the caption-position to bottom
  breaklines=true,                % sets automatic line breaking
  breakatwhitespace=true,         % sets if automatic breaks should only happen at whitespace
  belowcaptionskip=1\baselineskip,
  basicstyle=\footnotesize\ttfamily,
  keywordstyle=\bfseries\color{green!40!black},
  commentstyle=\itshape\color{purple!40!black},
  identifierstyle=\color{blue},
  backgroundcolor=\color{gray!10!white},
  title=\lstname,                 % show the filename of files included with \lstinputlisting;
}

\lstdefinestyle{customasm}{
    belowcaptionskip=1\baselineskip,
    frame=single, 
    frameround=tttt,
    xleftmargin=\parindent,
    language=[x86masm]Assembler,
    basicstyle=\footnotesize\ttfamily,
    commentstyle=\itshape\color{green!60!black},
    keywordstyle=\color{blue!80!black},
    identifierstyle=\color{red!80!black},
    tabsize=4,
    numbers=left,
    numbersep=8pt,
    stepnumber=1,
    numberstyle=\tiny\color{gray}, 
    columns = fullflexible,
    title=\lstname,                 % show the filename of files included with \lstinputlisting;
}
